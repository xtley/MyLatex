\documentclass[lang=cn,10pt]{elegantbook}
\usepackage{listings} % 引入 listings 包
\usepackage{xcolor} % 引入 xcolor 包,用于颜色定义
% 设置 listings 包的选项
\lstset{
    language=C++, % 代码语言
    basicstyle=\ttfamily, % 使用等宽字体
    keywordstyle=\color{blue}, % 关键词颜色
    commentstyle=\color{green}, % 注释颜色
    stringstyle=\color{red}, % 字符串颜色
    breaklines=true, % 自动换行
    numbers=left, % 行号在左侧显示
    numberstyle=\tiny, % 行号字体大小
    frame=single, % 代码框样式
    showstringspaces=false, % 不显示字符串中的空格标记
}


\title{计算机考研408:计算机组成原理}
\subtitle{YuTeng\LaTeX{}}

\author{Yu Teng}
\institute{和光同尘}
\date{December 29, 2024}
\version{1.0}

\extrainfo{不要以为抹消过去,重新来过,即可发生什么改变。—— 比企谷八幡}

\setcounter{tocdepth}{3}

\logo{logo-blue.png}
\cover{cover.jpg}

% 本文档命令
\usepackage{array}
\newcommand{\ccr}[1]{\makecell{{\color{#1}\rule{1cm}{1cm}}}}


\begin{document}

\maketitle
\frontmatter

\tableofcontents

\mainmatter

\chapter{计算机系统概述}

\section{计算机发展历程}

\subsection{计算机硬件的发展}

\begin{lstlisting}
    #include <iostream>
    
    int main() {
        std::cout << "Hello, World!" << std::endl;
        return 0;
    }
\end{lstlisting}

\chapter{数据的表示与运算}

\section{原码 反码 补码 移码}

\subsection{定义}
真值: 日常生活中使用的,带有+,-号的数字,如+15,-8,真值是机器数代表的实际值。 \\
机器数: 就真值的符号数字化,0表示正好,1表示负号,真值有原码,反码,补码等表示方法,如0101表示+5。\\
原码: 用机器数的最高位表示数的符号,其余各位数表示数的绝对值。\\
反码: 符号位与原码相同,真值为正数时,数值位与原码相同,真值为负数时,数值位与原码相反。\\
补码: 负数时=反码+1。\\
移码:

\subsection{相互转换}
正数时: 原码=反码=补码。\\
负数时(符号位为1): \\
原码$\rightarrow$反码: 除符号位各位取反。\\
原码$\rightarrow$补码:  ①原码除符号位取反+1或反码+1。\\
②快速计算方法:对原码非符号位从右往左扫描,右起第一个1及其右边的0保持不变,其余各位取反,符号位保持不变(补码$\rightarrow$原码同样适用)。\\
原码$\rightarrow$移码:

\subsection{表示范围}
以8为二进制数为例:0000 0000\\
0的个数:  ① 原码: 0000 0000 正0和 1000 0000负0。\\
② 反码: 0000 0000 和 1111 1111。\\
③ 补码: 0000。\\
④ 移码: 1000 0000。\\
范围:\\
	原码: $-127\sim+127,-(2^n-1) \sim 2^n-1$。\\
	反码:$-127\sim+127,-(2^n-1)\sim2^n-1$。\\
补码: $-128\sim127,2^n\sim2^n-1$。\\
移码:$-128\sim+127,2^n\sim2^n-1$。


\chapter{数据的表示与运算}





\end{document}
